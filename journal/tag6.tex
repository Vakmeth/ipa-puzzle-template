\section{Tag 6: 22.01.2025}
\begin{table}[H]
    \begin{tabular}{|L{0.4\textwidth}|C{60pt}|C{60pt}|C{60pt}|}
        \hline
        \rowcolor{puzzleblue}\color{white}Tätigkeiten & \color{white}Beteiligte \color{white}Personen & \color{white}Aufwand Geplant (std) & \color{white}Aufwand Effektiv (std) \\
        \hline
        Planning & Marc Egli & 1 & 1 \\
        \hline
        Persönliches Fazit & Marc Egli & 1 & 1 \\
        \hline
        Verzeichnis und Anhang & Marc Egli & 2 & 2 \\
        \hline
        Prüfung vor Abgabe & Marc Egli & 1 & 1 \\
        \hline
        \textbf{Total} & & 5 &5 \\
        \hline
    \end{tabular}
    \caption{Tätigkeiten Tag 6}
\end{table}

\subsection*{Tagesablauf}
Heute war der letzte Tag der IPA. Das Ziel: Die Finalisierung der IPA. Ich führte zuerst das Planning für den heutigen Tag und
den letzten Sprint durch. Danach startete ich direkt mit dem persönlichen Fazit welches ich wie geplant umsetzen konnte.
Danach startete ich mit dem Verzeichnis und Anhang, welcher die vollen 2 Stunden in Anspruch nahm. Nach diesen zwei Stunden muss ich nun 
nur noch die IPA komplett durchlesen und die kleinen Fehler darin korrigieren.

\subsection*{Hilfestellungen}

Keine 

\subsection*{Reflexion}
Trotz der knappen Zeit konnte ich meine Scrum Abläufe beibehalten und die IPA finalisieren. Die Planung von heute
ging perfekt auf, was für eine Weiterentwicklung meiner Schätzungsfähigkeit spricht. 

\subsubsection*{Was lief gut}
Die restlichen Sektionen konnten beschrieben und finalisiert werden.

\subsubsection*{Was lief weniger gut}
Heute lief alles nach Plan, weswegen ich in diesem Abschnitt nichts erwähnenswertes aufzählen kann.

\subsubsection*{Meine Erkenntnisse von heute}
Meine Erkenntnis von heute ist, dass es eine gute Entscheidung war mehr Zeit in die Umsetzung zu stecken. Für die 
Finalisierung hätte ich auf jeden Fall nicht die vollen acht Stunden in Anspruch genommen. Somit wurde die Zeit richtig investiert.

\subsection*{Nächste Schritte}
Als nächster und letzter Schritt werde ich die IPA nochmals gegenlesen und dann per Mail an meinen Ausbildner senden. 

\pagebreak
