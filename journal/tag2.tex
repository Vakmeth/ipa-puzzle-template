\section{Tag 2: 15.01.2024}
\begin{table}[H]
    \begin{tabular}{|L{0.4\textwidth}|C{60pt}|C{60pt}|C{60pt}|}
        \hline
        \rowcolor{puzzleblue}\color{white}Tätigkeiten & \color{white}Beteiligte \color{white}Personen & \color{white}Aufwand Geplant (std) & \color{white}Aufwand Effektiv (std) \\
        \hline
        Backup Konzept & Marc Egli & 0 & 0.5 \\
        \hline
        Projektaufbauorganisation & Marc Egli & 2 & 2.25 \\
        \hline
        Standards & Marc Egli & 0 & 0.25 \\
        \hline
        Datenschutzkonzept & Marc Egli & 0 & 0.5 \\
        \hline
        Planning & Marc Egli & 1 & 1 \\
        \hline
        Daily & Marc Egli, Daniel Illi & 0.25 & 0.5 \\
        \hline
        Arbeitsjournale & Marc Egli & 2 & 0.25 \\
        \hline
        \textbf{Total} & & 5.25 & 5.25 \\
        \hline
    \end{tabular}
    \caption{Tätigkeiten Tag 2}
\end{table}

\subsection*{Tagesablauf}
Heute konnte ich dank den Erkenntnissen von gestern schnell mit der Latex Dokumentation
vorankommen. Das Backup-Konzept konnte ich direkt abschliessen und habe dort sogar noch eine
Viertelstunde gespart. Diese brauchte ich wiederum für die Projektaufbauorganisation. Was hier länger gedauert hat,
war das erstellen der Diagramme. Ich wollte die Scrum Rollen und Rollenverteilung möglichst übersichtlich machen, was Zeit
kostete. Zuletzt waren die Arbeitsjournale geplant, hier habe ich einen Fehler in meiner Planung gemerkt. Ich habe angenommen, dass
ich die Arbeitsjournale noch anpassen müsste und wegen der fehlenden Latex-Erfahrung habe ich deswegen zwei Stunden eingeplant.

Allerdings hatten wir schon ein Template für das Arbeitsjournal im Projekt, deswegen hat sich diese Zeit auf 0 Stunden reduziert. Die
übrige Zeit habe ich dafür aufgewendet Nachbesserungen an der Dokumentation im Bereich Datenschutzkonzept und Standards zu machen. Zudem hat
das Daily auch länger gedauert, welches die nötige Zeit dann rausholen konnte. 

Zum Schluss des Tages habe ich den Sprintabschluss und das Planning für die Umsetzungsphase
gemacht. 

\newpage

\subsection*{Hilfestellungen}
\begin{itemize}
    \item Nils Rauch: Nachfrage Tool für Erstellung des Zeitplans
    \item Daniel Illi: Nachfrage Datenschutzkonzept
\end{itemize}

\subsection*{Reflexion}
Heute konnte ich sehr erfolgreich mit der Latex Dokumentation arbeiten. Auch das Planning
lief gut, ich denke ich habe nun eine saubere Planung für die Umsetzungsphase welche mir genug
Spielraum lässt. Der Fehler mit dem Zeitplan und Arbeitsjournal hat mir zwar Zeit in der Dokumentation gekostet,
allerdings ist es besser zu viel Zeit als zu wenig geschätzt zu haben. 

\subsubsection*{Was lief gut}
Die Arbeit mit Latex ging heute ohne Probleme voran und meine Effizienz war heute deutlich grösser 
als gestern.

\subsubsection*{Was lief weniger gut}
Der Fehler im Zeitplan mit den Arbeitsjournalen hat mich in der Planung durcheinandergebracht.
Ich habe die Zeiten nun korrekt im Zeitplan vermerkt, damit keine weiteren Probleme darunter entstehen.

\subsubsection*{Meine Erkenntnisse von heute}
Ich sollte vor der Eintragung in den Zeitplan prüfen, ob nicht schon Dokumente existieren welche mir
einen Teil der Arbeit abnehmen. Ist dies der Fall, wie bei meinen Arbeitsjournalen kann ich die Aufwandschätzung um 
ein Wesentliches reduzieren.

\newpage

\subsection*{Nächste Schritte}
Morgen werde ich mit der Umsetzung der IPA starten. Dabei werde ich zuerst die Einführung in das Hitobito
Projekt dokumentieren und dann direkt in die Konzeption für eine Filterlösung der Personenlisten und Abos starten.
Dies ist ein Schritt der mich zusätzlich motiviert, denn ich kann endlich etwas anderes machen als dokumentieren.

\pagebreak
