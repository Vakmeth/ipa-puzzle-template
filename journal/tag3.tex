\section{Tag 3: 16.01.2024}
\begin{table}[H]
    \begin{tabular}{|L{0.4\textwidth}|C{60pt}|C{60pt}|C{60pt}|}
        \hline
        \rowcolor{puzzleblue}\color{white}Tätigkeiten & \color{white}Beteiligte \color{white}Personen & \color{white}Aufwand Geplant (std) & \color{white}Aufwand Effektiv (std) \\
        \hline
        Einführung & Marc Egli & 0.5 & 0.25 \\
        \hline
        Ist-Zustand & Marc Egli & 2 & 2.5 \\
        \hline
        Soll-Zustand & Marc Egli & 0.5 & 0.25 \\
        \hline
        Persönliche Vorgehensziele & Marc Egli & 0.5 & 0.25 \\
        \hline
        Anforderungen & Marc Egli & 1 & 0.75 \\
        \hline
        Entwurf & Marc Egli, Daniel Illi & 3.5 & 4 \\
        \hline
        \textbf{Total} & & 8 & 8 \\
        \hline
    \end{tabular}
    \caption{Tätigkeiten Tag 3}
\end{table}

\subsection*{Tagesablauf}
Heute war ein sehr anstrengender aber produktiver Tag. Ich konnte zu beginn direkt die Einführung abschliessen. Hier Kommentaren
ich schnell durch und sparte etwa eine Viertelstunde. Der Ist-Zustand hat dann länger gedauert. Das lag daran das ich ein Sequenzdiagramm
für den Personenlistenfilter und den Abonnementenfilter entworfen habe und danach noch eine möglichst detaillierte Beschreibung der Komponenten lifern wollte.
Dafür konnte ich noch vor dem Mittag die Persönlichen Vorgehensziele und Anforderungen definieren. Am Nachmittag startete ich dann mit einem
Meeting mit Daniel Illi, meiner Verantwortlichen Fachkraft. Mir waren einige Fragen bezüglich dem Aufbau der Abo-Filter aufgekommen, welche ich mit ihm
klären konnte. Nach diesem Meeting konnte ich mit dem erlangten Wissen direkt in den Entwurf starten. Dieser hat mich mental am meisten beansprucht, da ich mich 
in die Filter-Thematik einarbeiten musste, um nachvollziehen zu können, wie diese aufgebaut sind. Den Entwurf konnte ich nun soweit bringen,
dass ich noch die Klassenstruktur der Abonnementenfilter erfassen und die Konzeption für eine Lösung machen muss.



\subsection*{Hilfestellungen}
\begin{itemize}
    \item Daniel Illi: Nachfrage Aufbau Abonnementenfilter
\end{itemize}

\subsection*{Reflexion}
Der heutige Tag war zwar produktiv und ich konnte viele Teile meines Zeitplanes abschliessen, jedoch muss ich unbedingt schneller im Dokumentieren werden.
Ich habe morgen noch 2h für den Entwurf und habe das Gefühl das diese Aufwanschätzung sehr knapp wird. Ich werde mich diesbezüglich morgen mit Daniel Illi unterhalten,
ob er mir Tipps zur Effizienzsteigerung hat. Meine grösste Angst gilt noch der Umsetzung welche morgen beginnt. Auch wenn ich ein Konzept in Sicht sehe, weiss ich auch das
die reine Featureentwicklung wohl meine grösste Schwäche ist, da ich diese bis jetzt selten im Hitobito einsetzen konnte.

\subsubsection*{Was lief gut}
Viele Sektionen in der Dokumentation konnten abgeschlossen werden. Einige davon konnten mit Diagrammen versehen werden, welches die Sektionen schon viel
klarer darstellt.

\subsubsection*{Was lief weniger gut}
Weniger gut lief meine Einschätzung der Geschwindigkeit meines Arbeitens. Ich muss schneller werden, ansonst wird es mit der Zeit knapp.

\subsubsection*{Meine Erkenntnisse von heute}
Meine Erkenntniss ist, dass ich schneller im Dokumentieren sein muss, allerdings die Qualität der Dokumentation gleichbehalte.

\subsection*{Nächste Schritte}
Morgen werde im Daily mit Daniel Illi besprechen, wie ich meine Effizienz im Dokumentieren steiger kann. Danach werde ich den Entwurf fertigstellen
und gegen den späten Morgen mit der Umsetzung meiner Arbeit beginnen. Ziel ist es morgen die Implementation grösstenteils fertigzustellen und ein funktionierendes
PoC zu haben. 

\pagebreak
