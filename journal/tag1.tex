\section{Tag 1: 14.01.2025}
\begin{table}[H]
    \begin{tabular}{|L{0.4\textwidth}|C{60pt}|C{60pt}|C{60pt}|}
        \hline
        \rowcolor{puzzleblue}\color{white}Tätigkeiten & \color{white}Beteiligte \color{white}Personen & \color{white}Aufwand Geplant (std) & \color{white}Aufwand Effektiv (std) \\
        \hline
        Planning & Marc Egli & 1 & 1 \\
        \hline
        Zeitplan & Marc Egli & 2 & 2 \\
        \hline
        Aufgabenstellung übernehmen & Marc Egli & 1 & 0.5 \\
        \hline
        Standards aus Github übernehmen & Marc Egli, Nils Rauch & 1 & 1.5 \\
        \hline
        IPA Schutzbedarfanalyse & Marc Egli, Nils Rauch, Olliver Brian, Olliver Dietschi, Thomas Ellenberg & 1 & 0.75 \\
        \hline
        Scrum Beschrieb & Marc Egli & 1 & 1.5 \\
        \hline
        Arbeitsjournal & Marc Egli & 0.25 & 0.5 \\
        \hline
        Backupkonzept & Marc Egli & 1 & 0.25 \\
        \hline
        \textbf{Total} &  & 8.25 & 8.25 \\
        \hline
    \end{tabular}
    \caption{Tätigkeiten Tag 1}
\end{table}

\subsection*{Tagesablauf}
Heute bin ich motiviert in die IPA gestartet. Als erstes habe ich am morgen nochmals die Spezifikationen für
die Dokumentation, durchgelesen und das Template für die IPA angepasst. Nachdem ich eine passende Struktur hatte, 
startete ich auch schon direkt mit dem ersten Sprint Planning dieser IPA. Dabei habe ich alle Tasks für den Sprint 1
im Backlog erfasst, diese dann im Refinement detaillierter Beschrieben und am Schluss in den Sprint Backlog geschoben.
Die ganze Planung habe ich mit Github Projects gemacht, leider kam ich da bezüglich Issues an die Grenzen denn leider kann mann 
diese nur definieren wenn die Issues einem Projekt, welches NICHT geforked ist, zugewiesen werden können. Dieses Problem werde
ich am Daily morgen mit meiner Fachkraft besprechen, evtl. weis er mehr dazu. 

Nach dem Planning begann ich mit dem Bereitstellen des Zeitplans. Ich übernahm das Tempalte welches ich ausgewählt hatte und
passte es auf meine drei Sprints in den kommenden zwei Wochen an. Zuerst dachte ich, dass ich den Zeitplan schneller fertigstellen könnte
jedoch hatte ich Probleme mit Google Sheets und das anlegen von gemergeden Spalten dauerte lange. Trotzdem ist die Planung aufgegangen und nach 2 Stunden
hatte ich einen geeigneten Zeitplan. 

Am Nachmittag Startete ich direkt mit dem Dokumentieren, angefangen bei den Standards unserer Firma. Es dauerete länger als gedacht,
alle Standards zu sammeln und in die Struktur der Dokumentation zu bringen, weswegen ich dort etwas Zeit verlor. Ein Teil davon konnte ich dann
bei der Schutzbedarfsanalyse wieder reinholen. Hier suchte ich den Kontakt mit anderen Mitarbeitern, um herauszufinden wo das Datenschutzkonzept für
Hitobito hinterlegt ist. Anscheinend wusste das Niemand aussert Oliver Brian, welcher mir dieses für die Ablage im Anhang zur Verfügung stellte.

Gegen den Ende des Tages habe ich die Projektmethode Scrum Beschrieben und dokumentiert wie ich mich während der IPA organisieren werde. 
Bezüglich der Aufteilung der Spalten der User Stories bin ich hier noch unsicher, ich werde dies sicher morgen am Daily auch mit
Daniel Illi abklären.

\subsection*{Hilfestellungen}
\begin{itemize}
    \item Oliver Brian: Nachfrage Datenschutzkonzept
    \item Nils Rauch: Nachfrage Sicherheitskonzept / Sicherheitsconventions Puzzle ITC
\end{itemize}

\subsection*{Reflexion}
Ich konnte heute schon einiges dokumentieren und habe nun eine Vorlage von der aus ich
einfach weiterarbeiten kann. Zusätzlich habe ich mit Github Project einen Ort an dem ich meinen
Fortschritt verwalte und mich selbst organisiere. Probleme gab es nur bei der Beschaffung des Datenschutzkonzeptes
und der Arbeit mit Google Sheets.

\subsubsection*{Was lief gut}
Grundsätzlich lief das Dokumentieren selbst sehr gut. Ich konnte alle restlichen Informationen
für die Standars oder die Projektmethode schnell Beschaffen und mich dann dem Dokumentieren widmen. 

\subsubsection*{Was lief weniger gut}
Weniger gut lief die Arbeit mit Google Sheets und die Arbeit mit der Latex Vorlage. Zum Teil hatte ich
recht lange bis ich herausfand wie ich eine Liste anlege oder ein Bild einfügen kann. Ausserdem habe ich mich im 
Zeitplan verschätzt und heute 9.25 anstatt 8.25 Stunden geschätzt, da ich im Google Sheets einen Fehler gemacht habe. Diesen konnte
ich aber schnell korrigieren, so dass ich heute auf geplante 8.25 Stunden komme, welche ich nun auch erreiche.

\subsubsection*{Meine Erkenntnisse von heute}
Mit erweitertem Latex know-how und dem Datenschutzkonzept in den Händen kann ich nun weiter dokumentieren.
Ich denke ich werde somit auch weniger Probleme mit Google Sheets und Latex haben, da ich heute schon viele
meiner Probleme lösen konnte.

\subsection*{Nächste Schritte}
Als nächstes werde ich morgen das Backupkonzept fertig machen und dann direkt zur Projektaufbauorganisation 
gehen. Nach Abschluss dieser Story kann ich den Sprint 1 Abschliessen und schon in den Sprint 2, der Konzeption / Umsetzung
starten.

\pagebreak
