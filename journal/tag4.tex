\section{Tag 4: 17.01.2024}
\begin{table}[H]
    \begin{tabular}{|L{0.4\textwidth}|C{60pt}|C{60pt}|C{60pt}|}
        \hline
        \rowcolor{puzzleblue}\color{white}Tätigkeiten & \color{white}Beteiligte \color{white}Personen & \color{white}Aufwand Geplant (std) & \color{white}Aufwand Effektiv (std) \\
        \hline
        Entwurf & Marc Egli, Daniel Illi & 2 & 8 \\
        \hline
        Umsetzung & Marc Egli & 6 & 0 \\
        \hline
        Arbeitsjournal & Marc Egli & 0.25 & 0.25 \\
        \textbf{Total} & & 8.25 & 8.25 \\
        \hline
    \end{tabular}
    \caption{Tätigkeiten Tag 4}
\end{table}

\subsection*{Tagesablauf}
Heute wollte ich nach Planung den Entwurf möglichst schnell abschliessen, jedoch war das Gegenteil der Fall.
Ich musste mich noch mehr in die Abläufe des Programmes momentan arbeiten und das Erstellen des Variantenentscheids dauerte
ewig. Zuletzt hat auch die Ausarbeitung enorm lange gebraucht. Das heisst ich muss am Dienstag umso mehr Gas geben um die Umsetzung abzuschliessen.
Damit ich mehr Zeit für diese gewinne, werde ich einen halben Tag der Finalisierung umbuchen und für die Umsetzung verwenden, wenn ich merke, dass
es nicht mehr reicht. Dies wird planmässig am Dienstag im Planning geschehen.

\subsection*{Hilfestellungen}
\begin{itemize}
    \item Daniel Illi: Nachfrage Aufbau des Konzeptes
\end{itemize}

\subsection*{Reflexion}
Obwohl ich sehr viel Zeit für den Entwurf gebrauch habe, bin ich dennoch froh, dass ich nun einen konkreten Plan für die Umsetzung habe.
So kann ich direkt am Dienstag mit dieser beginnen. Ich hätte vielleicht ein paar Teile des Entwurfes weniger detailliert machen müssen, auf der 
anderen Seite denke ich mir aber, dass ich bei der Umsetzung um genau diese Definition der Details sehr dankbar sein werde.

\subsubsection*{Was lief gut}
Nach der Dokumentation des Entwurfs verstehe ich nun genau wie der Abonnementenfilter und Personenlistenfilter aufgebaut sind. 
Ich habe ein Konzept welches eine Brücke zu den beiden Filterungsprozessen bildet und die alte Funktionalität immer noch gewährleistet,
was mich motiviert dieses schon bald umzusetzen.

\subsubsection*{Was lief weniger gut}
Trotz der guten Dokumentation ging wieder enorm viel Zeit verloren. Hier hätte ich mehr Zeit im Planning schätzen müssen und mindestens gleich viel Zeitplan
für den Entwurf wie die Umsetzung einrechnen müssen.

\subsubsection*{Meine Erkenntnisse von heute}
Meine Erkenntnisse liegen darin das ich weiss wie ich mein IPA Feature umsetzen will und das ich das nächste Mal im Planning wesentlich
mehr Zeit für den Entwurf einplane.

\subsection*{Nächste Schritte}
Meine nächsten Schritte sind das beginnen mit der Umsetzung meines definierten Konzepts.

\pagebreak
