\section{Tag 5: 21.01.2025}
\begin{table}[H]
    \begin{tabular}{|L{0.4\textwidth}|C{60pt}|C{60pt}|C{60pt}|}
        \hline
        \rowcolor{puzzleblue}\color{white}Tätigkeiten & \color{white}Beteiligte \color{white}Personen & \color{white}Aufwand Geplant (std) & \color{white}Aufwand Effektiv (std) \\
        \hline
        MVP umsetzten & Marc Egli & 2 & 8 \\
        \hline
        Aufgaben nach Sprint Planning 3 & Marc Egli & 6 & 0 \\
        \textbf{Total} & & 8 & 8 \\
        \hline
    \end{tabular}
    \caption{Tätigkeiten Tag 5}
\end{table}

\subsection*{Tagesablauf}
Am Morgen startete ich direkt mit der Umsetzung. Wie erwartet reichten die 2 Stunden nicht für die Umsetzung und ich musste das 
MVP Ticket in den 3 Sprint ziehen und so den Umsetzungsprint um weitere 6 Stunden erweitern. Dies lag vor allem dem Zeitaufwand
vom Entwurf letzter Woche zugrunde, da dieser massiv mehr Zeit benötigte als geplant war. Die Umsetzung selbst konnte ich schnell vorantreiben.
Mittels des erstellten Klassendiagrammes war ich nun sehr froh, dass ich mir die zusätzliche Zeit genommen habe, um die momentane Strukture der Filterprozesse
zu verstehen. Ich konnte die Anbindungen so direkt umsetzen und eine Klasse enwerfen welche
die Logik der Personnenlisten und Abonnemente verbindet. Nebst der Implementation selbst konnte ich mit den definierten manuellen Tests vom Testkonzept
sicherstellen das die Funktionalität immer noch wie bisher gewährleistet ist. 

\subsection*{Hilfestellungen}
Keine

\subsection*{Reflexion}
Wenn ich den heutigen Tag reflektiere bin ich ziemlich froh und auch ein klein wenig Stolz, dass ich die Implementation
so effizient umsetzen konnte. Ich habe ursprünglich mit mehr Problemen gerechnet, da ich wie bereits dokumentiert, noch nicht
viel Erfahrung in der Featureentwicklung gesammelt habe. Bestimmt gibt es noch bestimmte Punkte in der Implementation die optimiert werden können,
diese Aufgabe schreibe ich dem Zeitabschnitt vor der Einführung zu. Was ich noch besser machen müsste: Bestimmte Änderungen
am Konzept vorher erkennen. Ich habe während des Umsetztens gemerkt, dass ich den Filter sauberer mit ein paar Änderungen am ursprünglichen Entwurf 
umsetzen könnte. Diese Änderungen hätte ich vorher erkennen müssen. Ich werde in diesem Bereich sicher nochmals mit meiner verantwortlichen
Fachkraft besprechen, welche Methoden man bei der Konzeption hat, um genau solche Änderungen im Vorhinein zu planen.

\subsubsection*{Was lief gut}
Die Implementation selbst und das Coden lief sehr gut. Ich konnte endlich mal etwas umsetzten und musste nicht nur dokumentieren, was mich
zusätzlich motivierte. Auch das das Ergebnis am Schluss funktioniert hat, war ein Erfolgserlebnis von heute, ich rechnete unter anderem damit, dass ich
nur einen Teil der Funktion umsetzen könnte, aber da habe ich mich anscheinend geiirrt.

\subsubsection*{Was lief weniger gut}
Obwohl die Umsetzung erfolgreich von Statten ging, so habe ich dennoch 6 Stunden überschossen. Allerdings schreibe ich dies dem Entwurf zu,
nicht der Umsetzung, da die Umsetzung wie geplant 8 Stunden in Anspruch nahm.

\subsubsection*{Meine Erkenntnisse von heute}
Weniger negativ gegenüber der Umsetzung auftreten. Mit einer klaren Planung ist die Umsetzung nur ein weiterer Teil der IPA. 

\subsection*{Nächste Schritte}
Da ich mehr Zeit für den Umsetzungssprint verwendet habe, muss ich in der Finalisierung umso effizienter arbeiten. Das heisst: Keine unnötige Zeit
mit Latex-Syntax mehr verschwenden und direkt zum Punkt kommen.

\pagebreak
