\part[Projektdokumentation]{Projektdokumentation
                  \begin{center}
                     \begin{minipage}[c]{10.7cm}
                      \small Hitobito: Neue Generation von Personen-Filtern \\
                      Autor: Marc Egli
                     \end{minipage}
                  \end{center}
                 }
\chapter{Einführung}
Puzzle ITC ist ein schweizer Anbieter für Softwarelösungen. Die Firma hat ihren Hauptsitz in Bern,
besitzt aber weitere Standorte in Zürich, Luzern und Deutschland (Thüringen). Puzzle bietet als Unternehmen
die ganze Palette an IT-Services an, von Digital Transformation bis hin zu Data Analytics. Nebst den vielen Angeboten
tritt Puzzle dabei immer seine Grundwerte nach aussen, welche im Puzzlehouse abgebildet werden.

\begin{figure}[h]
   \centering
   \includegraphics[width=1\textwidth,]{puzzle-house.png}
   \caption{Rollen in Scrum}
\end{figure}

Hitobito ist eines der Angebote von Puzzle. Es ist ein Community-Management Tool und
als Open-Source Projekt auf Github zu finden. Das Tool wird von zahlreichen Verbänden, Parteien
und Organisationen verwendet und befindet sich darum in einer kontinuerilichen Weiterentwicklung. Mit dem Wagons-Gem
ermöglicht es Hitobito zudem spezielle Kundenanpassungen in einem eigenen "Wagon" zu vollziehen, ohne die Software anderer
Kunden mit-anzupassen.

Ich selbst arbeite jetzt seit einem halben Jahr im Hitobito und nahm darin vor allem Upgrades und Migrationen vor. So durfte ich
bspw. das Upgrade von RoR (Ruby on Rails) von 6.1 auf 7.1 vornehmen oder die Migration von MySQL auf Postgres vollziehen.

Da Hitobito von zahlreichen Kunden verwendet wird, ist die Applikation über die Jahre gewachsen. Viele Features wurden implementiert,
um sie schnell dem Kunden zur Verfügung zu stellen. Mit einem immer wachsenden Anforderungskatalog ergaben sich dadurch komplexe Arbeitsabläufe
welche im Tool etabliert wurden. Einer dieser komplexen Abläufe ist die Filterung nach Personen oder Abonnemente.

Mit dieser IPA soll die Filterung zwischen diesen zwei Entitäten homogenisiert werden. Um dies zu tun,
sollen zuerst zwei bis drei Konzepte ausgearbeitet und anschliessend in einem Variantenentscheid evaluiert werden. Für die Lösungsvariante wird in einem weiteren Schritt ein PoC (Proove of Concept) implementiert. 

Nach der IPA soll basierend auf der neuen Filterlogik ein neues UI entworfen werden, um nebst der Ordnung im Backend
eine besser User Experience für den Benutzer zu schaffen.

In einer Zeit in welcher Unternehmen mehr den je Wert auf ein sauberes Design und der User Experience von Webseiten und Applikationen geben, das auch
in einer älteren Applikation zu etablieren. Gerade bei einem Community-Management Tool wie Hitobito, welches tagtäglich von 
Personen bedient werden, welche nicht das technische Know-How dahinter besitzen, ist es wichtig Arbeitsabläufe so einfach wie möglich zu entwerfen, um 
maximale Effizienz für diese Personen zu garantieren. Durch eine Vereinfachung der Hitobito-Fitler machen wir damit einen ersten Schritt in die richtige
Richtung.

\chapter{Analyse}

\section{Ist-Zustand}

\section{Soll-Zustand}

\section{Persönliche Vorgehensziele}

\section{Anforderungen}
\subsection{Nicht funktionale Anforderungen}
 
\subsection{Funktionale Anforderungen}

\section{Abgrenzung}
 
\chapter{Entwurf}

\section{Lösungsvarianten}

\section{Variantenentscheid}

\section{Ausarbeitung}

\chapter{Ausführung}

\input{testprotokoll.tex}

\chapter{Einführung}

\chapter{Sprintabschlüsse}

\section{Abschluss Sprint Initialisierung}

\section{Abschluss Sprint Umsetzung}

\section{Abschluss Sprint Finalisierung}


